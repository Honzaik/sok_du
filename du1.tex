\documentclass[12pt, a4paper]{article}
\usepackage[utf8]{inputenc}
\usepackage{indentfirst} %indentace prvního odstavce
\usepackage{mathtools}
\usepackage{amsfonts}
\usepackage{amsmath}
\usepackage{amssymb}
\usepackage{graphicx}

\begin{document}

\section{}
Nechť \textbf{\textit{C}} je $[n,k,d]_{2}$ MDS kód, tedy $d=n-k+1$. Uvažujme generující matici tohoto kódu ve standardním tvaru a označme jí $C = [I_{k}|A]$, kde $A$ je matice z $\mathbb{F}^{k \times (n-k)}_{2}$. Matice $A$ je díky tomu, že \textbf{\textit{C}} je MDS, typu $\mathbb{F}^{k \times (n-k)}_{2} = \mathbb{F}^{k \times (d-1)}_{2}$.

%Pokud $k=n$, tak je to totální kód, a tudíž nikdy nemůže být 1-perfektní, jelikož 1-perfektní kód musí opravovat 1 chybu, tudíž $d > 2$, ale v případě totálního kódu $d=1$.

%Předpokládejme $k<n$, takže $d=n-k+1 \geq 2$. %

Pokud $k=1$, tak matice $C$ má pouze jeden řádek a aby byla splněna podmínka $d = n - 1 + 1$, tak v matici $A$ (která je vlastně řádkový vektor) musí být samé 1 (tedy i v $C$). Což znamená, že tento kód je triviální a je to opakovací kód. Tedy $[n,1,n]_{2}$ kód.

Předpokládejme nyní $k > 1$. V řádcích $C$ jsou kódová slova (prvky báze jsou také kódová slova), která musí splňovat, že jejich vzdálenost je $d$. Ale na prvních $k$ prvcích je jejich vzdálenost pouze 1, kvůli tomu, že $C$ obsahuje identickou matici, takže nezbývá nic jiného, než že v matici $A$ jsou samé jedničky, aby vzdálenost od nulového slova byla $d$. Pokud ale sečteme první 2 řádky matice $C$, tak z definice musí jejich součet být kódové slovo, ale to má právě 2 nenulové prvky, tedy musí platit $d \leq 2$.

Celkem tedy máme $1 \leq d \leq 2$. Takže $k = n-1$ nebo $k = n$. 

Jediné možné binární MDS kódy jsou tedy
\begin{itemize}
    \item $[n,1,n]_{2}$
    \item $[n,n,1]_{2}$
    \item $[n,n-1,2]_{2}$
\end{itemize}

My chceme ale 1-perfektní kódy, tedy $d > 2$. Tudíž jediný lineární kód co to splňuje je $[n,1,n]_{2}$, což je 2 prvkový opakovací kód $\textbf{\textit{C}}=\{000\dots0, 111\dots1\}$. Tento kód můžeme jednoduše upravit na nelineární přičtením jakéholiv nenulového vektoru k oboum vektorům této množiny.

Nelineární zbytek???
\end{document}