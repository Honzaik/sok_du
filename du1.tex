\documentclass[12pt, a4paper]{article}
\usepackage[utf8]{inputenc}
\usepackage{indentfirst} %indentace prvního odstavce
\usepackage{mathtools}
\usepackage{amsfonts}
\usepackage{amsmath}
\usepackage{amssymb}
\usepackage{graphicx}

\begin{document}

\section{}
Z přednášky víme, že je-li \textit{\textbf{C}} perfektní kód, tak je $r$-perfektní pro právě jedno $r=\frac{d-1}{2} \equiv d=2r+1$. Hledáme 1-perfektní kódy, tudíž $d=3$. Dále pro binární 1-perfektní kódy platí $V_{2}(n,1)=2^{n-k}$. \textit{\textbf{C}} je ale MDS, takže $n-k=d-1=2 \implies V_{2}(n,1)=2^{2}=4$. Také víme, že platí $V_{q}(n,r)=\sum_{i=0}^{r} \binom{n}{i}(q-1)^i$, tedy v našem případě $4=V_{2}(n,1)=\sum_{i=0}^{1} \binom{n}{i}=\binom{n}{0}+\binom{n}{1}=1+n \implies n = 3$. Hledáme tedy všechny kódy s parametry $n=3, k=n-d+1=3-3+1=1, d=3$. $k=1 \implies |\textit{\textbf{C}}|=2$. Všechny možné kódy jsou tedy:
\begin{itemize}
    \item $\textit{\textbf{C}}=\{000,111\}$, ten je jediný lineární.
    \item $\textit{\textbf{C}}=\{001,110\}$.
    \item $\textit{\textbf{C}}=\{010,101\}$.
    \item $\textit{\textbf{C}}=\{100,011\}$.
\end{itemize}


\section{}
Chceme najít nějaký $[8,4,4]_{2}$ kód. Abychom nejjednodušeji zaručili $d=4$, tak najdeme nejdříve kontrolní matici, která bude mít posloupnost libovolných 3 sloupců LN. Zvolme 
\[
H= \begin{pmatrix}
1 & 0 & 0 & 0 & 1 & 0 & 1 & 1 \\ 
0 & 1 & 0 & 0 & 1 & 1 & 1 & 0 \\ 
0 & 0 & 1 & 0 & 1 & 1 & 0 & 1 \\ 
0 & 0 & 0 & 1 & 0 & 1 & 1 & 1 
\end{pmatrix}
\]
Lze si jednoduše ověřit, že každá posloupnost 3 sloupců je LN. Ověření značně zjednodušuje to, že matice má v 1. polovině jednotkovou matici a také, že všechny sloupcové vektory mají váhu 1 nebo 3. Díky tomu lze jednoduše nahlédnout, že nulový sloupec matice neobsahuje ($d>1$), žádný sloupec se neopakuje ($d>2$), žádný sloupec není součtem jiných 2 ($d>3$). Naopak ale například součet 1., 2., 3. sloupců je 5. sloupec, takže $d<5 \implies d=4$. Zbylé parametry jsou zřejmé z velikosti matice. 

Máme tedy kontrolní matici $[8,4,4]_{2}$ kódu \textit{\textbf{C}}. Generující matici $C$ získáme díky tomu, že v generující matici kódu je báze Ker H. Tedy např.
\[
C= \begin{pmatrix}
1 & 1 & 1 & 0 & 1 & 0 & 0 & 0 \\ 
0 & 1 & 1 & 1 & 0 & 1 & 0 & 0 \\ 
1 & 1 & 0 & 1 & 0 & 0 & 1 & 0 \\ 
1 & 0 & 1 & 1 & 0 & 0 & 0 & 1 
\end{pmatrix}
\]

Nelineární kód stejných parametrů (nosnost = $\frac{k}{n} \implies k=4$) získáme například posunutím lineárního kódu (tedy vektorového podprostoru) \textit{\textbf{C}} o nějaký nenulový vektor. Například máme nelineární kód \textit{\textbf{D}} = \textit{\textbf{C}} + $e_1 = \{c + e_1 | c \in \textit{\textbf{C}}\}$. Posunutím jsme jistě délku ani počet prvků (k) nezměnili. Zbývá zkontrolovat zda se neporušila vlasnost $d=4$. To plyne z vlastností vzdálenosti vektorů. Nechť $x,y \in \textit{\textbf{D}} \implies \exists a,b \in \textit{\textbf{C}} : x=a+e_1, y=b+e_1 \implies d(x,y) = w(x-y) = w(a+e_1-(b-e_1))=w(a-b)=d(a,b)$. Tedy vzdálenost kódu \textit{\textbf{C}} = vzdálenosti kódu \textit{\textbf{D}}.

Máme-li kód \textit{\textbf{C}} s parametry $n,k,d>1$. Tak z definice propíchnutí kódu je zřejmé, že pro propíchnutý kód \textit{\textbf{C'}} platí že $n'=n-1$. Dále díky tomu, že původní kód měl vlastnost $d>1$, tak všechna slova mají rozdílné aspoň 2 souřadnice, takže ztracením jedné souřadnice jejich rovnost stále nemůže nastat. Tedy se parametr $k$ (velikost kódu) nezmění ($k'=k)$. Dále je zřejmé že $d'= d \vee d'=d-1$. Propíchnutý \textit{\textbf{C}} kód má tedy parametry $n=7,k=4,d=4$ nebo $n=7,k=4,d=3$. Vidíme, že $5>d>2$, takže jistě opravuje jednu chybu. Na to, aby byl 1 perfektní potřebujeme ještě zjistit zda-li $V_{2}(7,1)=2^{7-4}$, což plyne z vzorce pro $V_{q}(n,r)$ zmíněném výše.
\end{document}